 Torquato\cite{Sigmund1997} 
%%The energy factor defined as a ratio of the strain energy stored in
%the design cell to the strain energy stored in the whole structure due
%to work done by the stress difference results from the discontinuity at
%the interface.

Recently, based on the BESO method, a concurrent multiscale topology
design method with  stress-based material orientation optimization is
proposed to optimally design macrostructure, material microstructure and
material orientation distribution simultaneously \cite{Yan2019}. Yan et
al. \cite{Yan2020} proposed hybrid stress-strain method to determine the
material orientation for simultaneous design problem. They hybrid
stress-strain method is constructured by weighting the optimiality
condition of the mean compliance, in the stress and strain form. The
proposed method demostrates to deal with the shear weak and strong
materials and method can be extended to deal with 3D problem. However,
it is difficult to extend the analytical approach to complex loading
conditions or other physical situations, Hence from the prespective of
scalability, versatile orientation optimzation techniques using design
sensitivity are preferable.

CFAO: Bruyneeland \cite{Bruyneel2002} used sequential convex programming
to composite structure when considering the material orientations and
ply thickness simultaneously based on Methods of Moving Asymptotes
(MMA). Jiang et.al. \cite{Jia2008} modified the original SIMP method to
accommodate anisotropic behavior by including fiber angle as a design
variable, and applied to FRC for minimum compliance. The method is
commanly referred as solid orthotropic material penalization (SOMP).
\cite{Lindgaard2011} gradient based optimization of the buckling load of
laminate composite structures considering fiber angle deisng variables.
The optimization formualtion are based on either linear or geomterically
nonlinear analysis and formulated as mathematical programming problems
solved using gradient based techniques.  In [4], a three-dimensional
example was considered as a cantilever beam with a unit load applied at
the free end. This approach performed 3D topology optimization by
traversing sequentially to the orthogonal direction of a 3D structure
resulting in different compliance for each direction, which can be
useful for designers when considering the potential loading scenarios.
However, for composite materials with a strong anisotropy, the optimal
topology can be significantly different from the results obtained using
this approach. It demonstrates handling 2D composite components and very
low-resolution 3D problems, but the fiber orientation is still
constrained to a plane.
Level set methods offer a clear description of the design’s boundary
while optimization. In this framework, \cite{Wang2004, Wang2015} are the
adopted methods to design a structure that is composed of  phases of
materials with level set functions.


\subsubsection*{Sequentical}
\cite{Chen2021} design procedure integrate TO, Fiber placement technique
and CF4 to print high load-bearing FRC cantilever and three-point
bending specimens and their mechanical performance was comparatively
investigated. this study did not address the anisotropic behaviour of
continuous fibre-reinforced composites during the topological
optimisation, for potential further weight-saving.
\subsubsection*{Simultaneous}
\cite{Zhang2021} a TO framework for  hyperelastic structure with
nonlinear and anisotropic fiber reinforcement under large deformation.
the optimized distributiopn of fiber orientation is chosen from a srt of
discrete orientation defined a priori, and several fiber orientation
interploation schems are studied

\subsection*{Spatially varying fiber fraction}
CCF reinforced composites have been increasingly applied and
investigated for additive manufacturing techniques. For instance,
Sugiyama et al. \cite{Sugiyama2018} use CF4 techology to fabricate
sandwich structures with honeycomb, rhomboid, rectangular and circular
core shapes to study their mechanical properties under three-point
bending. Pyl et al. \cite{Pyl2019} demostrated the placement of
continuous constant fibres fraction to reduce stress concentrations
around the hole is exploited in open-hole tests. The experiential
demonstration shows that the reduction in the maximum value of stress
concentration factor \cite{Malakhov2016} and increment in the stiffness
\cite{Sugiyama2020} of the curvilinear geometrically discontinuous
structure by considering the design of fiber orientation and fiber
volume fraction simultaneously. However, it is more conducive to apply
topology optimisation in the first stage for designing optimal
engineering structures with specific aims such as light weight or
stiffness maximisation, but current design studies have mostly neglected
the topology optimisation stage and focused rather on fibre ply analysis
of simple structures (i.e. bolt-jointed panels with simple geometries).
Although there are theoretical exploitations for topology optimisation
of continuous fibre composites, they lack manufacturing applications and
experimental investigations. Hence, it is essential and vital to develop
multidisciplinary design and analysis which integrates topology
optimisation, carbon fibre reinforcement techniques and additive
manufacturing of fibre-reinforced composite parts.ù


Adopting similar orientation  variable  given in \cite{Nomura2015}, Lee
et al.\cite{Lee2018} presented a topology optimizationmethod for the
sequential three-step design of material layout and fibre orientation in
functionally graded fibre-reinforced composite structures.