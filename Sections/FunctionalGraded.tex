\section{Functionally graded anisotropic material}
Functionally graded composites \cite{Udupa2014} are inhomogeneous materials, consisting of two or more materials, engineered to have a continuously varying spatial composition and structure. Recent studies \cite{Fernandez2019,Jiang2019,Chandrasekhar2020} have shown that CF4 technology is ready to manufacture FRC structures with continuous yet spatially varying fiber paths and fiber volume fractions. Thus, if properly optimized, the spatial variation in FRC material properties may result in better performance than the fixed FRC material volume fraction. Therefore, a composite structure with FRC material in a fraction with voids and variable fiber density is termed a functionally graded anisotropic material. Furthermore, the before-mentioned gradation of FRC material brings considerably larger design freedom to design for additive manufacturing.  Accordingly, Li et al. \cite{Li2021} considered a SIMP-based sequential topology optimization approach to design functionally graded fiber-reinforced anisotropic composites by considering fiber fraction along with material fraction in a given design space. A sequential process begins with designing an isotropic-material matrix with voids, inserting fiber selectively, then optimally orienting the fiber. However, the approach sacrifices the exploration of new topologies that might be optimal for anisotropic composite materials. Therefore, the following works investigated the simultaneous design of isotropic-material matrix topology, fiber material layout, and orientation.