\section{Functionally graded anisotropic material}
Functionally graded composites \cite{Udupa2014} are inhomogeneous materials that consisting of two or more materials and are engineered to continuously vary the spatial composition and structure. Recent studies \cite{Fernandez2019,Jiang2019,Chandrasekhar2020} have shown that CF4 is ready to manufacture FRC structures with continuous yet spatially varying fibre paths and fibre volume fractions. Thus, if properly optimized, the spatial variation in FRC material properties may result in better performance than that for a fixed FRC material volume fraction. Therefore, a composite structure comprising a FRC material with voids and a variable fibre density is termed a functionally graded anisotropic material (FGRC). Furthermore, the aforementioned gradation of a FRC material provides a considerably larger design freedom to CF4. Accordingly, Li et al. \cite{Lee2018} considered a SIMP-based sequential TO approach to design FGRC by considering fiber fraction along with material fraction in a given design space. A sequential process begins with designing an isotropic-material matrix with voids, inserting fiber selectively, and then, optimally orienting the fibers. However, this approach sacrifices the exploration of new topologies that might be optimal for FGRC. Therefore, the following works investigated the simultaneous design of isotropic material matrix topology, fiber material layout, and orientation.

Desai et al. \cite{Desai2021} applied the topological derivative method to tailor a spatially varying fiber fraction. In addition, the dense arrangement of fibers was evenly spacing for the manufacturability of the part while retaining their specific patterns. However, the structural performance as a result of the simplification of the dense fiber arrangement was not evaluated, thus throwing into question the reliability of the printed part. In their work, a different fiber orientation approach was achieved by computing anisotropic topological derivatives in the polar coordinate system. A possible solution to avoid postprocessing for dense fiber arrangements with their technique might be to consider feature-based mapping methods.

The work, as mentioned earlier, implemented single scale approaches to optimize the distribution and orientation of the FRC material. However, CF4 also provides an effective means to fabricate mono-scale structures and multiscale structures. Thus, spatially varying material distributions and geometric patterns spanning at least two or more scales hold a promising future for designing next-generation lightweight structures. On the other hand, the multiscale strategy for anisotropic materials is challenging due to the following reasons: length scale controls, ability to produce models for fracture and damage criteria to capture actual anisotropic behaviour, and unique treatments at the boundaries of the domain, for example. These factors must be investigated through experiments or the use of appropriate numerical tools to estimate the actual performance of printed parts. Only a few works address the multiscale approach for FRCs based on the knowledge of the authors. Hence, interested readers can refer to Wu et al. \cite{Wu2021} review paper to understand the general framework for multiscale TO. Kim et al. \cite{Kim2020} adopted the homogenization method for designing spatially varying fiber volume fractions and fiber orientations, and simultaneously, the SIMP was used to design the macrostructure composite topology. Finally, the de-homogenization procedure \cite{Groen2018} applied to fiber microstructures obtained in the coarser mesh was visualized by projecting at a finer mesh. Various benchmark and multi-load structure problems have been studied and concluded that locally varying FRC materials augment the global stiffness to the structure more than a fixed fiber volume fraction or isotropic multi-material structure. In continuation of the Kim methodology, Jung \cite{Jung2022} proposes a 3D TO approach for designing a FGRC with spatially-varying fiber fractions and orientations. In conclusion, the multiscale framework has further enabled us to exploit the design freedom offered by CF4; however, no study to date fabricates and experimentally validates the results of a FGRC.