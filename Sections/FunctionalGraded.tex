\section{Functionally graded anisotropic material}
Functionally graded composites \cite{Udupa2014} are inhomogeneous materials, consisting of two or more materials, engineered to have a continuously varying spatial composition and structure. Recent studies \cite{Fernandez2019,Jiang2019,Chandrasekhar2020} have shown that CF4 technology is ready to manufacture FRC structures with continuous yet spatially varying fiber paths and fiber volume fractions. Thus, if properly optimized, the spatial variation in FRC material properties may result in better performance than the fixed FRC material volume fraction. Therefore, a composite structure with FRC material in a fraction with voids and variable fiber density is termed a functionally graded anisotropic material. Furthermore, the before-mentioned gradation of FRC material brings considerably larger design freedom to design for additive manufacturing.  Accordingly, Li et al. \cite{Lee2018} considered a SIMP-based sequential topology optimization approach to design functionally graded fiber-reinforced anisotropic composites by considering fiber fraction along with material fraction in a given design space. A sequential process begins with designing an isotropic-material matrix with voids, inserting fiber selectively, then optimally orienting the fiber. However, the approach sacrifices the exploration of new topologies that might be optimal for anisotropic composite materials. Therefore, the following works investigated the simultaneous design of isotropic-material matrix topology, fiber material layout, and orientation.

Desai et al. \cite{Desai2021} applied the topological derivative method for tailoring a spatially-varying fiber fraction. In addition, the dense arrangement of fibers is evenly spacing for the part's manufacturability while retaining their specific patterns. However, structural performance later in the simplification of the dense fiber arrangement is not evaluated, thus questioning the printed part's reliability. In their work, a differnet fiber orientation approach is achieved by computing anisotropic topological derivatives in the polar coordinate system. A possible solution to avoid postprocessing for dense fiber might be to consider the feature-based mapping methods with their technique.

The work, as mentioned earlier, implemented "single-scale" approaches optimizing the distribution and orientation of FRC material. However, AM also provides an effective means to fabricate mono-scale structures and multiscale structures. Thus, spatially-varying material distribution and geometric patterns spanning at least two-scale or more scales hold a promising future for designing next-generation lightweight structures. On the other hand, the multiscale strategy for anisotropic material is challenging due to the following reason: length scale controls, models for fracture and damage criteria to capture actual anisotropic behavior, unique treatments at the boundaries' of the domain, etc. to be investigated through experiments or using appropriate numerical tools for estimating the actual performance of printed parts. Based on the author's knowledge, only a few works address the multiscale approach for fiber-reinforced composites. Hence, interested readers can refer to Wu et al. \cite{Wu2021} review paper to understand the general framework for multiscale topology optimization.
Kim et al. \cite{Kim2020} adopted the homogenization method for designing spatially-varying fiber volume fraction and fiber orientation, and simultaneously, the SIMP designed the macrostructure composite topology. Finally, the de-homogenization procedure \cite{Groen2018} applied on fiber microstructures obtained in the coarser mesh is visualized by projecting at a finer mesh.  Various benchmark and multi-load structure problems are studied to conclude that locally varying FRC material further augments the global stiffness to the structure than the fixed fiber volume fraction or isotropic multi-material structure. In continuation of the Kim methodology, Jung \cite{Jung2022} proposes a 3-D topology optimization approach for designing a functionally graded composite structure with spatially-varying fiber fractions and orientations. In conclusion, the multiscale framework has further enabled us to exploit the design freedom offered by the CF4 technology; however, no study to date fabricates and experimentally validates attained functionally graded composite design results.