\section{Introduction}
CF4 technology allows fabrication of FRC material with the continuous
spatial in-plane variation of fiber angle and fiber volume fraction,
thus expanding design space as opposed to variable \cite{Ghiasi2009} and
constant stiffness laminate \cite{Ghiasi2009}. Moreover, CF4 technology
can achieve out-of-plane variation of fiber angle due to the
fiber-reinforced composite's self-supporting characteristics. Numerous
studies have shown that fiber orientation optimization can significantly
tailor structural performance such as stress
concentration\cite{Sugiyama2020}, stiffness \cite{Malakhov2016},
load-bearing capacity, buckling load, and the natural frequency
\cite{Zhang2011}. Therefore, the design of the FRC structures requires
optimization methods that reflect design freedom offered by CF4
technologies, including constraints, to thoroughly exploit the
anisotropic properties of FRC material \ cite{Xu2018}. Amidst the
ingredients of the optimization method for FRC structures with fiber
parameterization schemes and optimization algorithms have notable
influences on the quality of the solution.

\subsection{Paramterization schemes for fiber orientation}
The parameterization scheme implements a numerical description of fiber
orientation patterns and defines variables for the optimization. It
should ensure spatial continuity of fiber angle so that CF4 technology
can produce the structure, and it should also provide enough design
freedom so that optimization algorithm can consider more candidate
designs.

\subsubsection*{Continuous Parameterization}
The straightforward parametrization of fiber orientation design uses the
angle itself as the design variable. The design variable is then the
continuous and independent parameter at design points, for instance, at
centers of finite elements. Handling the continuous fiber orientation
design presents difficulties due to a fourth-order transform tensor,
which rotates the tensor to a given angle composed of multivalued sine
and cosine functions, rendering a non-convex optimization problem. Thus,
optimizing the fiber orientation is susceptible to the initial fiber
configuration and produces difficulties obtaining the optimized
solution. In addition, the lack of fiber continuity in a large region
puts a question mark on the realizability of the attained optimized
solution. As illustrated in \cite{Stegmann2005}, suboptimal solutions
\cite{Ghiasi2009, Xu2018} are the persistent outcome of a continuous
fiber orientation design problem. Various modifications in
parametrization schemes and different optimization algorithm methods
have been considered to handle the sub-optimal design issue for the
continuous fiber orientation problem --- one brute-force way to avoid it
by relaxing the design space. For instance, free material optimization
(FMO)\cite{Zowe1997, BenTal1999} further relaxed the design space by
independently parameterizing each stiffness tensor element as the design
variable. However, interpretation of the obtained stiffness tensor and
linking it to the feasible physical design makes this approach
challenging. Xia and Shi \cite{Xia2017}

An alternative is employing curvilinear parameterization schemes that define
fiber paths as the graphs of analytical function, which guarantee continuity of
fiber angle and have a small number of design variables \cite{Bruyneel2013,
Lemaire2015, Hao2018}.  Nevertheless, the restrictive design search space will
limit the tailorability of the fiber path, thus deteriorates the optimization
problem's stability \cite{Ghiasi2010} and quality of the optimized solution.
Adjectenly, the parameterization schemes can follow equidistant iso-contours of
a level set function to represent curvilinear fiber paths \cite{Brampton2015,
Papapetrou2020}, thus naturally ensuring fiber continuity and being often
parallel to the neighboring fiber paths. Furthermore, the optimization result
becomes highly dependent on the initial configuration, and local solutions often
appear \cite{Tian2021}.
\subsubsection*{Discrete Paramterization}
The counter approach reduces the design search space to avoid multiple local
minima issues where the optimized solution is highly sensitive to the initial
fiber configuration. Thus,  Stegmann and  Lund \cite{Stegmann2005,
Lindgaard2011} stretched the design search space by choosing discrete fiber
orientation candidates, define apriori, which parameterized the design space
into the discrete material candidates. These transversely isotropic material
models are defined for different fiber orientations for the same isotropic
elasticity tensor and share some similarities with the multi-material
optimization problem in \cite{Bendsoee1999, Yin2001}. The suggested approach
assigned weighting functions to different candidates and employed gradient-based
optimization and penalization coefficient to force the weighting functions
towards either zero or one to seek fiber convergence, i.e., one discrete
material at each design point. This method is known as Discrete Material
optimization (DMO). DMO laid the foundation for SFP (Shape Function and with
Penalization) \cite{Bruyneel2011}, BCP (Bi-value Coding Parameterization)
\cite{Gao2012} to perform discrete fiber orientation optimization . A comparison
using various numerical examples on these methodologies on discrete fiber
orientation optimization drawn in \cite{Kiyono2017}. Contrarily to CFAO, it does
not cover design problems for continuously varying orientation distributions.
Thus, it permits a limited scope to fully exploit the potential of modern
technology's continuously varying orientation in composites. \cite{ArianNik2012,
Malakhov2016, Sugiyama2020}. Secondly, these methods fail to address the fiber
convergence even against the significant penalization factor; hence, their
benefit relies on an optimization algorithm to circumvent impractical mixtures
of fiber angles. Third, the discrete parameterization scheme should further
minimize the number of design candidates for efficient optimization.

Kiyono et al. \cite{Kiyono2017} proposed a continuation of the
computational approach suggested by Yin and Ananthasuresh \cite{Yin2001}
by utilizing the normal distribution function to guarantee fiber
convergence, low sensitivity to the initial fiber configuration, and the
continuity of the fiber orientation Salas et al. \cite{Salas2018}
proposed a Self-Penalization Interpolation Model for Fiber Orientation
(SPIMFO) based on convergent Talyor series for sine and cosine functions
and optimized the dynamic design of laminated piezo composite actuators
combining SPIMFO with SIMP method. Nomura et al. \cite{Nomura2015}.
proposed a general methodology for the simultaneous design of structural
topology and continuous and discrete material orientation with the
isoparametric projection method. In that study, the Cartesian components
of the orientation vector were chosen as the orientation design
variables. The 2$\pi$ ambiguity stems from the periodic nature of the
angular representation is addressed to yields better control on
manufacturability. However, the simultaneous design will significantly
change the orientations with a change in topology, potentially causing
LOSS.


\subsection*{Optimization Algorithm}
\subsubsection*{Analytical approaches}
Analytical methods determined early studies of optimal fibre orientation
that dates back to the pioneering work of Pedersen on strain-based
method \cite{Pedersen1989, Pedersen1990, Pedersen1991} and followed by
the stress-based method.  In the strain-based method, analytically
derived equations study the sensitivity of strain energy with respect to
material orientation for orthotropic material based on the assumption
that the strain energy is invariant on material orientation. A similar
analytical deduction can be derived for the stress-based method that
except constant stress field with respect to material orientation. These
methods eliminate the need for iteration during the design process and
significantly increase computational efficiency compared with
mathematical programming. In addition, the stress-based method produces
a slightly stiffer structure than the strain method because strong
couplings exist among the orientational variables when the strain field
is used. In both scenarios, optimization proceeds towards convergence.
The material orientation tends to coincide with the major principal
stress/strain fields for relatively 'weak' shear and some shear 'strong'
types of orthotropic materials. However, the methods are highly
dependent on the initial fibre configuration, and the solution of both
approaches will fail when the local and repeated global minima (more
than one solution has the same global minimum value) occurs
\cite{Gea2004}. Nevertheless, these methods form the basis for future
research on material orientation optimization for fibre reinforced
composite materials. These methods' shortcomings were due to the
assumption of constant stress or strain fields that was removed by the
energy-based method introduced by Luo and Gea \cite{Diaz1992,
Cheng1994}. The proposed hybrid framework estimates the strain fields'
and stress fields' dependency on fibre orientation by introducing an
approximate energy factor. Their research showed that an energy-based
method produces numerically more optimal solutions to stress-based and
strain-based methods. The method use inclusion model is used to analyze
the variations of the strain and stress of one design cell due to the
rotation of orthotropic material in that design cell. It assumes that
the displacements are continuous at the interface of the design cell
after the cell orientation changes. Thus, the strain of the design cell
will not change because the cell is under the same displacement boundary
condition. In contrast, the stress of the design cell will vary, suffer
from stress discontinuity at the interface. Furthermore, the assumption
to consider additional stress and strain fields for both the design cell
and its surrounding resolves stress discontinuity caused at the
interface. By introducing the concept of energy factor, Luo et al.
evaluated additional stress and strain field.

\subsubsection*{Gradient-based approaches}
\cite{Lindgaard2011} gradient based optimization of the  buckling load of
laminate composite structures considering fiber angle deisng variables. The
optimization formualtion are based on either linear or geomterically nonlinear
analysis and formulated as mathematical programming problems solved using
gradient based techniques. Xia and Shi \cite{Xia2018} First, a hierarchy of
parameterizations are constructed. At each level of the hierarchy, one has a
distinct parameterization for the fiber angle arrangement. From the top level to
the bottom level of the hierarchy, the number and density of design points for
the Shepard interpolation increase, thus the design freedom and the resolution
of parameterization increase as well. Second, at each level of the hierarchy, an
optimization problem is formulated, and these optimization problems are solved
successively. After the optimization problem at a coarse level is solved, one
goes to its neighboring finer level, using the solution of the former one to
compute an initial design for the latter one. Again, the Shepard interpolation
is used for the computation of initial design.


formulated cascadic multilevel optimization problem,
where there results from coarse level serves as an initial design for finer
level. 
\subsubsection*{Heurstic-based approaches}
Another outlook on approaching the LOSS problem is using an optimization
solver better at global searching ability, such as the genetic algorithm
\cite{Wang2020} and simulated annealing algorithm \cite{Hasancebi2010}.
The heuristic algorithm can find "global minimum" and allow handling a
discrete variable, but this always sacrifices the computational cost.
Sigmund \cite{Sigmund2011} questions the usefulness of non-gradient
approaches in TO, showing the inefficiency of non-gradient
algorithm-based optimization problems with many design variables and
constraints. As a result, gradient-based solver i.e. Method of Moving
Asymptotes (MMA) \cite{Svanberg1987} set the framework for the TO
problem and has been pushed significantly to avoid the LOSS
\cite{Shen2020}.
 