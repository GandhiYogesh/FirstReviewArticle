

\section{Paramterization schemes for fiber orientation}
The parameterization scheme implements a numerical description of fiber orientation patterns and defines variables for the optimization. It should ensure spatial continuity of fiber angles so that CF4 technology can produce the structure. It should also provide enough design freedom so that the optimization algorithm can consider more candidate designs. This section discusses various parameterizations schemes used in the literature to optimize fiber-reinforced composite structures.

\subsection{Continuous parameterization}
The continuous parametrization of fiber orientation (CFO) design uses the angle itself as the design variable \cite{Bruyneel2002, Lindgaard2011}. The design variable is the continuous and independent parameter that provides flexibility in changing the orientation across the design points, relaxing orientation design space. However, handling a continuous fiber orientation design presents difficulties due to a fourth-order transform tensor that rotates to a given angle composed of multivalued sine and cosine functions, resulting in a non-convex optimization problem. Furthermore, optimizing the fiber orientation is susceptible to the initial fiber configuration, thus causing difficulties obtaining the optimized solution. As illustrated in \cite{Stegmann2005}, suboptimal solutions are the persistent outcome of a continuous fiber orientation design problem. One brute-force way to avoid it is by further relaxing the design space. For instance, free material optimization (FMO)\cite{Zowe1997, BenTal1999} parameterizing each stiffness tensor element independently as the design variable. This secures the scheme from the complexity of the orientation design variable of the design space. However, as compensation, point-wise nonlinear constraints ensure the positive semi-definitiveness of the obtained stiffness tensor and link it to the feasible physical design, making this approach challenging. Nomura et al. \cite {Nomura2019} formulated orientation design variable as a tensor field to simplify the first tensor invariant constraint and remove nonlinear constraints successfully introduced due to the second tensor invariant. Still, as commented, the violation of these constraints is observed at the joint point of the structural members where the orientation shows the discontinuous distribution.



% Gradient-based algorithms are efficient in finding local minima in spaces of any number of dimensions with nonlinear constraints. The method employs variational methods to form an unconstrained optimization problem to satisfy structure behavior's criteria resulting in two unknown design variables and Lagrange multipliers, which are iteratively updated to fulfill the optimality criteria at every iteration.


% Frequent calculations of the objective function and constraints and their derivatives in nonlinear mathematical programming methods become expensive and time-consuming with the increase in the design variables. Moreover, they are proportional to the resolution of the discretized domain that represents a limiting factor for the capability of a structural optimization algorithm. Therefore, an alternative is the optimality criteria instead of directly optimizing the objective function. This method employs variational methods to form an unconstrained optimization problem to satisfy structure behavior's criteria resulting in two unknown design variables and Lagrange multipliers, which are iteratively updated to fulfill the optimality criteria at every iteration

For anisotropic materials, early studies utilizing the analytically derived optimally criterion \cite{M.C.E1904} for optimizing fiber orientation dates back to the pioneering work of Pedersen on strain-based method \cite{Pedersen1989, Pedersen1990, Pedersen1991}. In that work, strain energy density was transformed into principal strain, and it was concluded that material orientation axes that lie along principal strain axes always give stationary energy density. However, Cheng \cite{Cheng1994} argued that the discussion is limited to a unit cell case where the orientation variable is separated from the design domain to obtain extreme strain energy. After that, a similar deduction using iterative optimality criteria \cite{Zhou1992, Zhou1993} formulated the stress-based method \cite{Suzuki1991} by exercising an invariant stress field for material orientation. Finally, Diaz and Bendsoe \cite{Diaz1992} extended the stress-based method to determine the optimal orientation optimization problem corresponding to multiple loads. Despite their similarity, the stress-based method produces a slightly stiffer structure than the strain method because strong couplings exist among the orientational variables when the strain field is used \cite{Cheng1994}. Conclusively, Gea and Luo \cite{Gea2004} demonstrated that the fiber orientation coincides with the principal stress/strain fields for relatively weak shear and some strong shear types of anisotropic materials.

Further, the methods are highly dependent on the initial fiber configuration, and both approaches will fail for shear 'strong' type materials due to repeated global minimum solutions. Nevertheless, these methods form the basis for future research on material orientation optimization for FRC materials. The shortcomings of these methods encouraged the formulation of the energy-based method introduced by Luo and Gea \cite{Luo1998a, Luo1998}. This method uses an inclusion cell to estimate the dependency of the strain fields and stress fields on the fiber orientation by introducing an approximate energy factor. Yet, the dependence of energy factors on the traction stress, material properties, and direction of the inclusion cell and its surroundings make it challenging to formulate the framework for 3D and complex loading problems. Following the principles of the energy-based method, Yan et al. \cite{Yan2020} proposed a hybrid stress-strain method by weighting the optimality condition of the mean compliance in the stress and strain form. Numerical examples demonstrate their method on weak and strong shear materials and extension to 3D problems. The assumption regarding the elemental strain and stress field invariant to the neighboring elemental orientation is considered; however, it may restrict the solution of 3D problems and result in a suboptimal solution.


An alternative is employing curvilinear parameterization schemes that define fiber paths as the graphs of analytical function, which guarantee continuity of fiber angle and have a small number of design variables \cite{Bruyneel2013, Lemaire2015, Hao2018}. Nevertheless, the restrictive design search space will limit the tailoring of the fiber path, thus deteriorating the stability of the optimization problem  \cite{Ghiasi2010} and quality of the optimized solution. Also, the parameterization schemes can follow equidistant iso-contours of a level set function to represent curvilinear fiber paths \cite{Brampton2015, Papapetrou2020}, naturally ensuring fiber continuity and being often parallel to the neighboring fiber paths. Furthermore, the optimization result becomes highly dependent on the initial configuration, and local solutions often appear \cite{Tian2021}.

\subsection{Discrete paramterization}
The counter scheme reduces the orientation design space to avoid multiple local optima issues, where the optimized solution is highly sensitive to the initial fiber configuration. Therefore, a discrete orientation optimization formulation was solved using a genetic algorithm at the cost of a computational burden \cite{Riche1993} \cite{Nagendra1996} \cite{Liu2000_haftka}. Lund \cite{Stegmann2005} relaxes the combinatorial problem to a continuous optimization problem. By choosing discrete fiber orientation candidates, which are defined a priori, the orientation design space is parametrized into discrete material candidates. These transversely isotropic material models are defined for different fiber orientations for the same isotropic elasticity tensor. Thus the scheme share some similarities with the multi-material optimization problem in \cite{Bendsoee1999, Yin2001}. The suggested scheme assigned weighting functions to different candidates and employed gradient-based optimization with penalization coefficient, forcing the weighting functions to seek a binary design and fiber convergence, i.e., one discrete material at each design point. This method is known as Discrete Material optimization (DMO). DMO laid the foundation for Shape function and with penalization (SFP) \cite{Bruyneel2011}, bi-value coding parametrization (BCP) \cite{Gao2012} to perform discrete fiber orientation optimization. A comparison for these methodologies that use various numerical examples is contained in \cite{Kiyono2017}.

DMO does not incorporate design problems for continuously varying orientation distributions. First, it is an imperative design consideration to circumvent stress constraints and a degradation in the strength by an order of magnitude compared to that for continuous fibre paths due to fibre discontinuity.  Consequently, it permits a limited scope to fully exploit the potential of modern technology's continuously varying orientation in composites \cite{ArianNik2012, Malakhov2016, Sugiyama2020}. Secondly, these methods fail to address the fiber convergence even against the significant penalization factor; hence, their benefit relies on an optimization algorithm to circumvent impractical mixtures of fiber orientations. Third, the discrete parametrization schemes should further minimize the number of design candidates for efficient optimization. Concerning these drawbacks, Kiyono et al. \cite{Kiyono2017} proposed a parametrizing scheme that continues the computational approach suggested by Yin and Ananthasuresh \cite{Yin2001}. Introducing a normal distribution function as a weighting function in their parametrizing scheme guarantees fiber convergence, a low sensitivity to the initial fiber configuration, and continuity of the fiber orientation. Another different work proposed a self-penalization interpolation model for fiber orientation (SPIMFO)  based on convergent Talyor series for sine and cosine functions to optimized composite hyperelastic material \cite{Silva2021} and the dynamic design of laminated piezo-composite actuators \cite{Salas2018}.

\subsection{Hybrid paramterization}
Utilizing continuous and discrete methodology benefits is another alternative to fiber orientation optimization. The key idea in the following approaches is to fill the gaps by acknowledging the beneficial characteristics of both strategies to improve computational efficiency and/or reduce local optima and/or resolve fiber continuity and/or manufacturability issues. Therefore, an approach to reduce the risk of falling into local optimal without sacrificing the fiber continuity can use both discrete and continuous parametrization as suggested by Luo et al. \cite{Luo2020}. Their work proposed a coarse-to-fine strategy, where the orientation design space is divided into discrete sub-intervals. After that, the CFO searches for an optimized solution in a sub-interval, where the sub-interval selection problem is solved using the DMO approach. However, no criterion is defined to determine the number of sub-intervals required in advance.

Nevertheless, the proposed strategy provides flexibility to integrate alternatives that are suggested for DMO and CFO approaches. Nomura et al. \cite{Nomura2015}  studied the cartesian system for orientation design variables to improve initial design dependency and local optima issues encountered in the continuous parameterization approach. The parametrization scheme was further extended to yield an optimized discrete orientation design for a given discrete orientation set in their work. Moreover, the characteristics representing the orientation design variables into the vectorial form consider the 2$\pi$ ambiguity, which occurs due to the periodic nature of the orientation design variable. Introducing vectorial design variable as a point-wise quadratic inequality constraint yields more interpolated elasticity tensor than the single variable polar representation.

Xia and Shi \cite{Xia2017}  develops a continuous global function by applying the shepherd interpolation method at scattered design points to represent the fiber orientation throughout the design domain. The benefit of the interpolation function is that it ensure fiber continuity while considering a reduced orientation design space in contrast with CFO. Unfortunately, it suffers from the initial configuration and ends at the sub-optimal solution. Another work of Xia \cite{Xia2018} applied multilevel optimization for fiber orientation optimization and verified its efficiency against the single-level optimization. Still, the optimization results in different fiber arrangements for different initial fiber orientations. As a result, the efficiency of the multilevel approach relies on the attained fiber orientation field at a coarse level since the optimization at the successive refined level starts from an initial design computed at its neighboring coarser level.

\subsection{Feature-based parameterization}
The parameterization, as mentioned earlier, introduces low-level fiber material representations, such as pixel or voxel-based, thus representing the designs with variables proportional to the number of pixels or voxels in the design space. Moreover, these techniques render organic and free-form designs, which require sophisticated postprocessing to distinguish fiber paths for the use of CF4. Therefore, to avail manufacturable solutions with designs with few variables, fiber material can be considered as a geometric feature with high-level parameters. High-level parameters refer to spatial dimensions associated with the size, position, or orientation of a feature. Finally, feature-mapping techniques map these features onto a fixed mesh for analysis; an extensive review of feature-mapping methods by Wein et al. \cite{Wein2020} details the components of feature-mapping techniques and discusses their implementation in structural optimization.
Geometry projection \cite{Norato2015} is a feature-mapping technique extended  to represent the design via cylindrical bars reinforced with continuous fibers \cite{Smith2021} and performs the analysis using a fixed finite element mesh. The interpolation of the material properties at the junction of multiple bars made of an anisotropic material is penalized as a convex combination of the penalized effective densities for each component. It demonstrates the method can easily integrate shape constraints on the structural form offered by CF4. Thus, this work introduces the groundwork for using the geometry projection method for fiber-orientation optimization design problems.

\begin{figure}[!ht]
    \centering
    \includegraphics[width=0.4\textwidth]{./../Schematics/GTO_FRC.tex}
    \caption{Schematics for hybrid geometry projection method for FRC}
\end{figure}