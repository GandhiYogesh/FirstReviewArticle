\section{Closure}
This work presents a brief review of TO methods for FRC structures applied to CF4. First, the study underlines the single-scale TO approaches that simultaneously or sequentially design fiber orientation and structural's topology. It further reports the usefulness of multiscale TO for realizing FRC and extending it for FGRC structures. Moreover, it highlights emerging TO methodologies such as feature-mapping and multi-component are suitable choices to attain ready-to-manufacture design and to introduce manufacturing constraints with ease.

The TO methods can attain continuous fiber orientation by incorporating continuous or discrete fiber angle or both as a design variable in the numerical optimization scheme. However, continuous strategies are susceptible to the initial fiber configuration due to the problem's nonconvexity and restrictive design space. On the other hand, discrete strategy solves the problem's nonconvexity using the multi-material method but increases the number of design variables. Therefore, a hybrid implementation that couples both strategies serve as a promising strategy to avoid earlier mentioned issues, but an efficient algorithm still needs to be addressed. In addition, the recent use of the geometry projection method for anisotropic materials is new attention in the fiber orientation optimization methods.

Overall, continuous parameterization schemes are considered for spatially varying fiber orientation and/or fiber volume fraction. In addition, these methods are amenable to CF4 and multi-axis material deposition technology. However, realizing part using continuous strategies may require path planning techniques to account for manufacturing constraints, whereas discrete strategies clearly illustrate the fiber paths in the design domain; hence parts can avoid further postprocessing. Therefore, DMO is widely chosen in the automotive and aerospace industries due to manufacturability ease. On the other hand, feature-mapping methods provide ready-to-manufacture parts; however, it still requires validation for complex problems, such as bucking stability and eigenvalue analysis. Hence, the choice of the TO methods for designing complex FRC parts applied to AM requires various considerations; thus, this review highlights the different aspects of TO methods used for FRC structures to laid an essential foundation for the researchers entering the field of the TO of FRC material.