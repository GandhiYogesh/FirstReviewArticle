\section{A brief note on optimization algorithms in TO}
Reasonably arranging the fiber orientation is critical to effectively handling an anisotropic material, which is vital for designing next-generation lightweight composite structures. Frequently, fiber orientation optimization creates a difficulty associated with local optima and discontinuous functions. Thus, to address this, gradient-free algorithms \cite{Hasancebi2010} are more qualified because of their global searching ability \cite{Reuschel1999, Voelkl2018}. Furthermore, by allowing differentiable functions, mixed design variables, and discrete space, introduce a relaxed formulation that has the advantage of obtaining fewer local optima. However, the inefficiency of most gradient-free algorithms requires numerous function evaluations, which is impractical for expensive finite element simulations. Hence,  adoption of gradient-based algorithms, i.e., Optimality Criteria Method (OCM), Method of Moving Asymptotes (MMA) \cite{Svanberg1987}, and Sequential Linear Programming (SLP) \cite{Dunning2015} become a reasonable choice for the TO problems.

The OCM is derived using the Lagrange function, which is composed of objective and constraint functions that satisfy the Karush-Kuhn-Tucker (KKT) condition for an optimal solution. The OCM procedure has double loops, where the inner loop updates the design variable, and the outer loop updates the Lagrange multiplier based on the KKT condition. However, the method cannot handle multiple constraints because the coupling of the Lagrange multiplier and the design variables requires solving a nonlinear equation. Therefore, Shen et al. \cite {Shen2020} questioned the lack of understanding about the orientation optimization algorithm to handle arbitrary constraints and loads in the OCM. A step length scheme for orientation optimization is advised to achieve global descent by normalizing the gradient vector and introducing a parameter to control the magnitude of material orientation in each iteration. However, the verification lacks the effect of adding constraints in the orientation optimization problem on the update scheme, a critical factor for the OCM. Thus, a more generalized OCM for the topology optimization of an anisotropic material is demanded from scalability and multiloading situations. Recently, Kim et al. \cite{Kim2021} interpreted the work of Patnaik et al. \cite{Patnaik1995} on parametric optimization and proposed a generalized optimality criteria method for topology optimization problems. The approach eliminates the compulsion to satisfy the constraints during every optimization iteration but should be met upon convergence.

On the other hand, SLP and MMA are general-purpose optimization strategies supporting various engineering multi-objective and multi-constraints nonlinear problems (NLP). In these first-order methods, the gradient information about a design point approximates the constraint and objective functions. In particular, for MMA, a hybrid form of the linear and reciprocal approximation \cite{Fuchs1990} has the advantage of being convex, which introduced the term convex linearization (CONLIN) \cite{Fleury_1989CONLIN} for approximating the optimization problem. Svanberg introduces a convex approximation variation that stabilizes and speeds up the convergence of process optimization by controlling moving asymptotes while the approximation remains convex and first-order. Furthermore, because the subproblem is separable and convex,  a dual approach or a primal-dual interior-point method can efficiently solve the NLP. However,  the reciprocal approximation in MMA might eliminate the linearity of approximation; therefore, SLP can serve as a suitable candidate unless convexity is not needed \cite{Barthelemy1993}.
