The cost-effective commercially available Additive Manufacturing (AM) or 3D Printing (3DP) technologies eliminate many limitations that previously plagued the manufacturing of highly tailored structural performance for multifunctional \cite{Ye2020} and multi-physics \cite{Luo2019_Multi} applications. Moreover, the short metamorphosis of AM technologies offers unique capabilities to realize the next-generation lightweight structure have brought great application potentials in several major industries such as aerospace \cite{Kokkinis2015, Berrocal2019}, automotive \cite{Wu2017} and medical\cite{Cramer2017}. First, AM's unique ability to fabricate a highly complex shape without a substantial increase in fabrication costs, along with the benefit s of reducing manufacturing preparation time, renders these technologies worthy investment for large-scale industries. Moreover, it offers the fabrication of lattice structures-a lightweight design comparative to the solid-filled parts. Thus, offer diversification of design to answers multifunctional material requirements in weight reduction, \cite{Fleck2010}, their ability to dissipate energy \cite{Qiao2008, Maskery2017}, heat \cite{Aremu2017} and vibration \cite{Cheng2018}. The printed polymer parts frequently consist of carbon nanotubes and short fibre to upgrade their mechanical performance.  Still, it cannot outperform \cite{Parandoush2017, Sano2018}, the mechanical strength offered by continuous fibre-reinforced composite laminate manufactured using conventional manufacturing tools. Hence, the shortcomings of 3D printed polymer composites aggravated the demand to develop Continuous Fiber Filament Fabrication (CF4) technology. CF4 technology provides a unique opportunity to reduce part distortion warping and support structures during printing, and fibre tension prevents nozzle clogging, a constant lookout with the polymer AM. Additionally, controlling the anisotropic properties of the fibre-reinforced composites can effectively distribute the loads throughout the laminate to maximize the structure's strength and stiffness.

CF4 technology allows fabrication of FRC material with the continuous spatial in-plane variation of fiber angle and fiber volume fraction, thus expanding design space as opposed to variable \cite{Ghiasi2009} and constant stiffness laminate \cite{Ghiasi2009}. Moreover, CF4 technology can achieve out-of-plane variation of fiber angle due to the fiber-reinforced composite's self-supporting characteristics. Numerous studies have shown that fiber orientation optimization can significantly tailor structural performance such as stress concentration\cite{Sugiyama2020}, stiffness \cite{Malakhov2016}, load-bearing capacity, buckling load, and the natural frequency \cite{Zhang2011}. Therefore, the design of the FRC structures requires optimization methods that reflect design freedom offered by CF4 technologies, including constraints, to thoroughly exploit the anisotropic properties of FRC material \cite{Xu2018}. Recognizing these potentials of CF4, returned a resurgent interest in utilizing the design strategies to exploit AM's performance-driven manufacturing technologies towards enhancing printed parts' overall functional performance. In contrast, to the geometric-driven or/and cost-driven manufacturing of components. The concept of performance-driven manufacturing is known as Design for Additive Manufacturing (DfAM) \cite{Plocher2019}.

Topology Optimization (TopOpt), one of the DfAM methods, is an iterative design tool to optimize a quantifiable objective while being intended to sustains loads, constraints, and boundary conditions. Topology Optimization is frequently adopted to design structurally sound parts and has subsequentially surpassed optimization design tools such as shape and size optimization in isolation. The seminal work of Bendsøe and Kikuchi \cite{Bendsoee1988} introducing the concept of TopOpt on the homogenization method; since then, TopOpt has been developed rapidly. TopOpt approaches can be summarized as follows: the homogenization method \cite{Bendsoee1988}; the Solid Isotropic Material with Penalization (SIMP) method \cite{Bendsoee1989, Rozvany1992}, the level set method \cite{Wang2003, Allaire2004}, the Evolutionary Structural Optimization (ESO) method \cite{Xie1993}; Topology Derivatives and Phase Field. The details of these approaches are discussed in the review papers \cite{Rozvany2009, VanDijk2013, Deaton2014} and some emerging TopOpt methods for smooth boundary representation include the 'Metamorphic Development Method' (MDM) \cite{Liu2000}, and the 'Moving Morphable Method' (MMM) \cite{Liu2017}. The general architecture of TopOpt starts with the definition of maximizing or minimizing a single or multi-target-objective function to fulfil a set of constraints such as volume, displacement, or frequency \cite{Li2019}. Then, as part of an iterative process, design variables, Finite Element Analysis (FEA), sensitivity analysis, regularization, and optimization steps are repeated in this order until convergence is achieved \cite{bendsoee2013book}.

The optimization concept applied to FRC materials allows finding the optimized material distribution, the optimized orientation of fiber paths, as well as optimized geometric contours of the laminate. Hence, the optimization method for FRC structures with continuous fiber parameterization schemes and optimization algorithms have notable influences on the quality of the solution. The article on the optimization of topology and fiber path orientation and thus, only related works are reviewed.



