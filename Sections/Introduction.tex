Cost-effective commercially avaiable additive manufacturing (AM) or 3D Printing (3DP) technologies eliminate many limitations that previously plagued the manufacturing of highly tailored structural performance for multi-functional \cite{Ye2020} and multi-physics \cite{Luo2019_Multi} applications.
Moreover, AM offers unique capabilities to realize the next-generation lightweight structure have brought great application potentials in several major industries such as aerospace \cite{Kokkinis2015, Berrocal2019}, automotive \cite{Wu2017} and medical\cite{Cramer2017}. First, AM techniques have the unique ability to fabricate highly complex shapes without a substantial increase in fabrication costs; also, the benefit of reducing manufacturing preparation time renders these technologies viable for large-scale industries. Moreover, it offers lattice structures, which are lightweight compared to solid parts. Thus, AM offers weight reduction \cite{Fleck2010} and the ability to dissipate energy \cite{Qiao2008, Maskery2017}, heat \cite{Aremu2017} and vibrations \cite{Cheng2018}. Printed polymer parts frequently consist of carbon nanotubes and short fibre to upgrade their mechanical performance.  Still, printed parts cannot outperform \cite{Parandoush2017, Sano2018}, the mechanical strength offered by continuous fibre-reinforced composite laminate manufactured using conventional manufacturing tools. Hence, the shortcomings of 3D printed polymer composites support the development of continuous fibre filament fabrication (CF4).  CF4 provides a unique opportunity to reduce part distortion warping and support structures during printing, and fibre tension prevents nozzle clogging, a constant challenge with polymer AM techniques. Additionally, controlling the anisotropic properties of FRCs can effectively distribute the loads throughout the laminate to maximize the strength and stiffness of the fabricated structures.

CF4 technology allows fabrication of FRC material with the continuous spatial in-plane variation of fiber angle and fiber volume fraction, thus expanding the design space compared to that for variable \cite{Ghiasi2010} and constant stiffness laminate \cite{Ghiasi2009}. Moreover, CF4 technology can achieve out-of-plane variation of fiber angle due to the fiber-reinforced composite's self-supporting characteristics of FRCs. Numerous studies have shown that fiber orientation optimization can significantly tailor structural performance such as stress concentration\cite{Sugiyama2020}, stiffness \cite{Malakhov2016}, buckling load, and the natural frequency \cite{Zhang2011}. Therefore, the design of the FRC structures requires optimization methods that reflect design freedom offered by CF4 technologies, including constraints, to thoroughly exploit the anisotropic properties of FRC material \cite{Xu2018}. These advantages of CF4 have resulted in an interest in utilizing design strategies to enhance the overall functional performance of printed parts. This is in contrast to the geometric-driven or/and cost-driven manufacturing of components. The concept of performance-driven manufacturing is known as Design for Additive Manufacturing (DfAM) \cite{Plocher2019}.

Topology optimization (TO), one of the DfAM methods, is an iterative design tool to optimize a quantifiable objective while being able to sustain loads, constraints, and boundary conditions. TO is frequently adopted to design structurally sound parts and has subsequentially surpassed design tools, such as shape and size optimization, in isolation.  The seminal work of Bendsøe and Kikuchi \cite{Bendsoee1988} introduced the concept of TO for the homogenization method; since then, TO has developed rapidly. TO approaches can be summarized as follows: the homogenization method \cite{Bendsoee1988}, the Solid Isotropic Material with Penalization (SIMP) method \cite{Bendsoee1989, Rozvany1992}, the level set method \cite{Wang2003, Allaire2004}, the Evolutionary Structural Optimization (ESO) method \cite{Xie1993}, and Phase Field \cite{Bourdin2003}. The details of these approaches are discussed in the review papers \cite{Rozvany2009, VanDijk2013, Deaton2014} and some emerging TO methods for smooth boundary representation include the 'Metamorphic Development Method' (MDM) \cite{Liu2000}, and the 'Moving Morphable Method' (MMM) \cite{Liu2017}. The general architecture of TO starts with the definition of maximizing or minimizing a single or multi-target-objective function to fulfil a set of constraints such as volume, displacement, or frequency \cite{Li2019}. Then, as part of an iterative process, design variables, finite element methods (FEMs), sensitivity analysis, regularization, and optimization steps are repeated in this order until convergence is achieved \cite{bendsoee2013book}.

The optimization concept applied to FRC materials enables the optimal material distribution, optimized orientation of fibre paths, and optimized geometric contours of a laminate to be found. Hence, the optimization method for FRC structures with continuous fibre parameterization schemes and algorithms have notable influences on the quality of the solution. The article is a brief review of topology and fibre path orientation optimization of FRCs, and thus, only related works are reviewed.



