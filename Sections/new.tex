\section{Topology Optimization for continuum structures}
\subsection{Problem Statement}
A general form of topology optimization
\subsection{Discrete Topology Optimization} Topology optimization aims to obtain binary solutions representing optimal structural layouts. Therefore, the topology optimization problem is a binary problem representing the void and solid regions of the structure, respectively. The well-known discrete topology optimization method is the Bi-directional Evolutionary Structural Optimization (BESO). Interested readers find comprehensive reviews on the BESO methods in \cite{Munk2019, Xia2018BESO}.  Another outlook on approaching the discrete problem is using a genetic algorithm \cite{Wang2020} that can find "global minimum" and allow handling a discrete variable, but this always sacrifices the computational cost. Furthermore, Sigmund \cite{Sigmund2011} questions the usefulness of non-gradient approaches in Topology Optimization. Finally, Sivapuram et al. \cite{Sivapuram2018} combined the features of BESO and the sequential integer linear programming for discrete topology optimization.
The discrete TopOpt uses binary design variables, one for each of the finite elements in the mesh of the structural domain. The design variable $1$ implies the finite element is filled with material, $0$ indicates void. Consider an objective function $\Phi(\rho, \mathrm{U}(\rho))$, constrained by $g_{i}\leq g_{i}^{*}.i\in[1,M]$, with $\rho$ being the design variables, where $N$ number of finite elements in mesh. The discrete variable optimization problem can be formulated as follows:
\begin{equation}
    \begin{aligned}
        \min _{\rho}   & : \Phi(\rho, \mathrm{U}(\rho))                                                         \\
        \text { s.t. } & : \sum_{e=1}^{N} v_{e} \rho_{e}=\mathbf{v}^{T} \mathbf{\rho} \leq V^{*}                \\
                       & : g_{i}(\mathbf{\rho}, \mathbf{U}(\mathbf{\rho})) \leq g_{i}^{*}, \quad i=1, \ldots, M \\
                       & : \rho_{e}=\left\{\begin{array}{l}
            0(\text { void }) \\
            1(\text { material })
        \end{array}, e=1, \ldots, N\right.                     \\
                       & : \mathbf{K}(\mathbf{\rho}) \mathbf{U}=\mathbf{F}
    \end{aligned}
\end{equation}
where $\mathbf{K}$ and $\mathbf{U}$ is the assembled stiffness matrix and displacement vector corresponding to finite elements in the mesh.
\subsection{Continuum Topology Optimization}
Density-based topology optimization is a broadly received idea considered topology optimization of continuum structures by using continuous density design variables that transform the binary variable optimization problem into a density distribution problem. The design variable can take any value from 0 to 1 such that $\rho \in [0,1]$. Thus, in such methods \cite{Bendsoee1988,Bendsoee1989}, optimized solutions do not explicitly exhibit structural boundaries, which challenges solving problems where explicit boundary identification is essential, e.g., in design-dependent and multiphysics problems. The continuous variable optimization problem can be formulated as follows:
\begin{equation}
    \begin{aligned}
        \min _{\rho}   & : \Phi(\rho, \mathrm{U}(\rho))                                                         \\
        \text { s.t. } & : \sum_{e=1}^{N} v_{e} \rho_{e}=\mathbf{v}^{T} \mathbf{\rho} \leq V^{*}                \\
                       & : g_{i}(\mathbf{\rho}, \mathbf{U}(\mathbf{\rho})) \leq g_{i}^{*}, \quad i=1, \ldots, M \\
                       & : 0 \leq\rho_{min}\leq\rho\leq 1                                                       \\
                       & : \mathbf{K}(\mathbf{\rho}) \mathbf{U}=\mathbf{F}
    \end{aligned}
\end{equation}