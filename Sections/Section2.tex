\section{Topology Optimization for continuum structures}
TO optimizes material layout within a given design space $\Omega$ for a given set of loads $\{F^{T}, \Gamma_{t}\}$, boundary conditions $\Gamma_{u}$, and constraints to maximize the system's performance. It uses FEMs to evaluate the design performance, and the design is optimized using either gradient-based mathematical programming techniques or non-gradient-based algorithms.
\begin{figure}[!ht]
    \centering
    \includegraphics[width=0.4\textwidth]{./../Schematics/Topopt}
    \caption{Density-based TO framework}
    \label{fig:Gantt}
\end{figure}

\subsection{Problem Statement}
A general  form of topology optimization can be written as an optimization problem:
\begin{equation}
    \begin{aligned}
        \min _{\rho} & : \Phi(\rho, \mathbf{U}(\rho))                                                         \\
                     & := \int_{\Omega}f(\rho, \mathbf{U}(\rho))                                              \\
        \text { s.t. }
                     & : G_{i}(\mathbf{\rho}, \mathbf{U}(\mathbf{\rho})) \leq G_{i}^{*}, \quad i=1, \ldots, Q \\
    \end{aligned}
\end{equation}
An objective function $\Phi$ represents the quantity being minimized or maximized to to maximize the performance of a system. The density of the material in each location as a design variable $\rho$ for the optimization problem.  The constraints $G_{i}$ are characteristics that the solution must satisfy. The FEM evaluates the field $\mathbf{U}$ that satisfies a linear or nonlinear state equation, since these equations do not have a known analytical solution. The design space $\Omega$ defines the optimizable volume $V^{*}$ for the design to exist, and the volume in the design space that the optimizer cannot modify is considered non-optimizable volume $\Gamma_s$. A characterisitic function $\chi$ is defined to descirbe the material domain $\Omega_{d}$ to be optimized:
\begin{equation}
    \chi(\mathbf{x})=
    \begin{cases}
        0, & \forall \mathbf{x} \in \Omega\backslash\Omega_{d} \\
        1, & \forall \mathbf{x} \in \Omega_{d}
    \end{cases}
\end{equation}

where $\mathbf{x}$ stands for a design point in $\Omega \  \text{and} \ \chi(\mathbf{x})$ is defined by a scalar function $\phi$ and the Heaviside function $H$ such that:
\begin{equation}\label{characterisitic function}
    \chi(\mathbf{x})=H(\phi(\mathbf{x})) =
    \begin{cases}
        0, & \forall \mathbf{x} \in \Omega\backslash\Omega_{d} \\
        1, & \forall \mathbf{x} \in \Omega_{d}
    \end{cases}
\end{equation}
To eliminate checkerboard patterns therefore generating mesh-independent results, the Helmholtz PDE filter \cite{Lazarov2011} is introduced to regularize $\phi$:
\begin{equation}\label{PDE filter}
    -R_{\phi}^{2}\nabla^{2}\tilde{\phi} + \tilde{\phi}=\phi
\end{equation}
where $R_{\phi}$ is the filter radius, and $\tilde{\phi}$ is the filtered field. Then the density field $\rho$ can be defined by an additional smoothed Heaviside function $\tilde{H}$:
\begin{equation}\label{density field}
    \rho = \tilde{H}(\tilde{\phi})
\end{equation}
After the series of regularization from $\phi \ \text{to} \ \rho $, the resulting density field is bounded between $0 \ \text{and} \ 1$. The general architecture of the topology optimization framework is depicted in the schematic.

The following broader categorization of the implementation has been used to solve topology optimization problems.

\subsection{Discrete Topology Optimization} The discrete variable optimization problem can be formulated as follows:
\begin{equation}
    \begin{aligned}
        \min _{\rho}   & : \Phi(\rho, \mathbf{U}(\rho))                                                         \\
        \text { s.t. } & : \sum_{e=1}^{N} v_{e} \rho_{e}=\mathbf{v}^{T} \mathbf{\rho} \leq V^{*}                \\
                       & : g_{i}(\mathbf{\rho}, \mathbf{U}(\mathbf{\rho})) \leq g_{i}^{*}, \quad i=1, \ldots, M \\
                       & : \rho_{e}=\left\{\begin{array}{l}
            0(\text { void }) \\
            1(\text { material })
        \end{array}, e=1, \ldots, N\right.                     \\
                       & : \mathbf{K}(\mathbf{\rho}) \mathbf{U}=\mathbf{F}
    \end{aligned}
\end{equation}
The TO problem is a binary problem representing the void and solid regions of the structure. The discrete TO uses binary design variables, one for each of the finite elements in the mesh of the structural domain. The design variable $1$ implies the finite element is filled with material, $0$ indicates void. For example, consider an objective function $\Phi(\rho, \mathbf{U}(\rho))$, constrained by $g_{i}\leq g_{i}^{*}.i\in[1,M]$, with $\rho$ being the design variables, where $N$ number of finite elements, $\mathbf{K}$ and $\mathbf{U}$ is the assembled stiffness matrix and displacement vector corresponding to finite elements in the mesh.

The well-known discrete topology optimization method is the Bi-directional Evolutionary Structural Optimization (BESO). Interested readers find comprehensive reviews on the BESO methods in \cite{Munk2019, Xia2018BESO}. Another outlook on approaching the discrete problem is using a genetic algorithm \cite{Wang2020} that can find "global minimum" and allow handling a discrete variable, but this always sacrifices the computational cost. Furthermore, Sigmund \cite{Sigmund2011} questions the usefulness of non-gradient approaches in TO. Recently, Sivapuram et al. \cite{Sivapuram2018} combined the features of BESO and the sequential integer linear programming for discrete topology optimization.


\subsection{Continuous Topology Optimization}
The continuous variable optimization problem can be formulated as follows:
\begin{equation}
    \begin{aligned}
        \min _{\rho}   & : \Phi(\rho, \mathbf{U}(\rho))                                                         \\
        \text { s.t. } & : \sum_{e=1}^{N} v_{e} \rho_{e}=\mathbf{v}^{T} \mathbf{\rho} \leq V^{*}                \\
                       & : g_{i}(\mathbf{\rho}, \mathbf{U}(\mathbf{\rho})) \leq g_{i}^{*}, \quad i=1, \ldots, M \\
                       & : 0 \leq\rho_{min}\leq\rho\leq 1                                                       \\
                       & : \mathbf{K}(\mathbf{\rho}) \mathbf{U}=\mathbf{F}
    \end{aligned}
\end{equation}
Density-based topology optimization is a broadly received idea in the TO of continuum structures by using continuous density design variables that transform the binary variable optimization problem into a density distribution problem. The design variable can take any value from 0 to 1 such that $\rho \in [0,1]$; thus, the material properties across the elements are varied. Therefore, material properties are interpolated using a power function, where the intermediate properties are penalized, thus forcing the design back to the binary structure.

This transformation enables the use of gradient-based information; unfortunately, the penalization parameter also introduces nonconvexities; hence, there is a high risk of falling into local minima. Furthermore, in SIMP, optimized solutions do not explicitly exhibit structural boundaries, which provides a challenge when solving problems where explicit boundary identification is essential, e.g., in design-dependent and multiphysics  problems. Thus, TO can be formulated in the nodal variables \cite{Belytschko2003} that control an implicit function description of the shape to address the exact boundary identification problem.