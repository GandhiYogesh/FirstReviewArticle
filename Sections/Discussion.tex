\section{Discussion and future trends}
The discussion focuses on the suitability of given TO for anisotropic materials given a pre-requisite understanding of the manufacturing process and its limitations. Therefore, the following discussion does not address composite manufacturing and its differences to adopt the particular TO method.

The manufacturing design freedom extended by available composite manufacturing processes provides the flexibility to develop and integrate TO approaches for designing anisotropic material orientations \cite{Ferreira2019}. Therefore, the paradigm of performance-driven design focuses on investigating the suitability of TO methods that can fully exploit the design freedom offered by manufacturing technologies. Thus, the existing techniques for material orientation are categorized into four major classes, as stated previously.

The optimization of a prescribed set of alternative discrete angles, which is referred to as the DMO, is often preferred in the aerospace, automotive, and wind turbine industries for manufacturability reasons. The DMO  approach is favourable for composite laminate designs  \cite{Hvejsel2011} \cite{Kennedy2013} \cite{Lund2018}  because a mixed-integer programming problem is formulated as a continuous problem that can be solved efficiently using gradient-based optimizers. As a result, substantial problems that might not be amenable to gradient-free methods can use DMO parameterization. An indirect approach is to apply lamination parameters, as introduced by Tsai and Pagano \cite{Tsai1968}. However, despite these methods' popularity, they are limited to a prescribed set of alternative discrete angles, while the CF4 processes have higher freedom in orientation control can produce higher performing composites.

Therefore, the continuous orientation methods are most suitable for CF4 processes. Furthermore, these methods provide the highest freedom in terms of shape and variable stiffness. Thus, the continuous orientation formulation directs the material deposition path planning to ensure the fiber trajectory curvature, fiber continuity, fiber fraction, and offset distance between adjacent fibers, unlike discrete methods where the fiber convergence and fiber continuity are challenging to attain. Papapetrou et al. \cite{Papapetrou2020} designed the topology and material orientation in parts simultaneously; the optimized results were post-processed using continuous fiber path planning to ensure realizability. A sequential scheme was proposed \cite{Chen2021, Wang2021pp} where the fiber placement that was based on load transmission follows isotropic TO; this is contrary to Liu \cite{Liu2017pp}, who adopted concurrent fiber path planning and structural TO. The multi-axis material deposition technology using a robotic arm requires an extension of the TO algorithm to envelop the 3D fiber orientation, in contrast to in-plane printing. Schmidt et al. \cite{Schmidt2020} introduced azimuth and elevation angles to extend the CFO method for 3D fiber orientation. In addition, they emphasized the issues of nonconvexity of the compliance and sensitivity to the initial fiber orientations by investigating the orientation parameter space to mitigate the problems\cite{Kubalak2020}. Finally, the realizability of 3D printed composite is studied by  Fedulov et al. \cite{Fedulov2021}, where they proposed a filtering technique for fast convergence.

Utilizing TO methods for exploring the CF4, generally speaking, heighten the composite manufacturing cost, especially when committing these technologies for large-scale structure parts. Therefore, understanding the trade-off among commercial aspects, i.e., realizability, practicability, and structural design, requires assimilating the benefits of the discrete, continuous, and multi-component methodology. Thus, a hybrid parameterization scheme optimizes the structural topology and material orientation, including multi-component optimization (MTO) that decomposes product geometry while guaranteeing manufacturing constraints that might significantly impact the quality and cost of the end product. Initially, a genetic algorithm was used to solve MTO \cite{Lyu2005}, and then recently, a gradient-based optimization algorithm was used by Zhou et al.\cite{Zhou2018_MTO}. Zhou et al. \cite{Zhou2018} further extended their work for structures made of multiple composite components with tailored material orientations, without a prescribed set of alternative discrete angles. Therefore, this method can produce regions fabricated separately and joined with either continuous or discrete material orientation methods.

Feature-based parametrization follows the ideology of ready-to-manufacturability with a necessary restriction on the spatial distribution of the fiber orientations. It envelops commercial aspects for the realizability of composite parts by introducing CAD-based features to ease the manufacturing process with the potential for layerwise design. Moreover, it further simplifies the design space by reducing the design variables considerably. It is noted that published work only considered stiffness-driven design. However, it is also critical to consider failure modes for composite parts manifested using the layerwise AM process. Incorporating these failure criteria renders markedly different designs that raise the method's relevance in fabricating FRC structures.

