\section{Discussion}
The discussion focuses on the suitability of given topology optimization for anisotropic material, given a pre-requisite understanding of the manufacturing process and its limitations. Therefore, the following discussion does not address composite manufacturing technology and its differences to adopt the particular topology optimization method.

The manufacturing design freedom extended by available composite manufacturing processes provides the flexibility to develop and integrate topology optimization approaches for designing anisotropic material orientations \cite{Ferreira2019}. Therefore, the paradigm of performance-driven design focuses on investigating the suitability of topology optimization methods that can fully exploit the design freedom offered by manufacturing technology. Thus, the existing techniques for material orientation are broadcasted into four major classes, as stated previously.

Optimizing a prescribed set of alternative discrete angles, named the discrete orientation method, is often preferred in aerospace, automotive, and wind turbine industries for manufacturability reasons. DMO approach is favorable for composite laminate design \cite{Hvejsel2011} \cite{Kennedy2013} \cite{Lund2018}  because a mixed-integer programming problem is formulated as a continuous problem that can be solved efficiently using gradient-based optimizers. As a result, substantial problems that might not be amenable to gradient-free methods can use DMO parameterization. An indirect approach is applying lamination parameters, as Tsai and Pagano introduced \cite{Tsai1968}. However, despite these methods' popularity, they are limited to a prescribed set of alternative discrete angles, while the CF4 processes have higher freedom in orientation control can produce higher performing composites.

Continuous orientation methods provide the highest freedom in shape and variable stiffness. Thus, the continuous orientation formulation directs material deposition path planning to ensure fiber trajectory curvature, fiber continuity, fiber fraction, and offset distance between adjacent fiber deposition, unlike discrete methods where fiber convergence and fiber continuity is difficult to attain. For example, Papapetrou et al. \cite{Papapetrou2020}  designed part's topology and material orientation simultaneously; the optimized results were post-processed using continuous fiber path planning to ensure realizability. Furthermore, a sequential scheme is proposed in another work \cite{Chen2021, Wang2021pp} where fiber placement based on load transmission follows isotropic topology optimization, contrary to Liu \cite{Liu2017pp}, who adopted concurrent fiber path planning and structural topology optimization. The multi-axis material deposition technology using the robotic arm requires an extension of the topology optimization algorithm to envelop the 3D fiber orientation in contrast to in-plane printing. Schmidt et al. \cite{Schmidt2020} introduced azimuth and elevation angles to extend the CFAO method for 3D fiber orientation. In addition, they emphasized the issues of nonconvexity of the compliance and sensitivity to the initial fiber orientation guess—a study by investigating the orientation parameter space to mitigate the issues \cite{Kubalak2020}. Finally, the realizability of 3D printed composite is studied by  Fedulov et al. \cite{Fedulov2021}, where they proposed a filtering technique for fast convergence.

Utilizing topology optimization methods for exploring the CF4 process, generally speaking, heighten the composite manufacturing cost, especially when committing these technologies for large-scale structure parts. Therefore, understanding the trade-off among commercial aspects, i.e., realizability, practicability, and structural design, further pushes on assimilating the benefits of the discrete, continuous and multi-component methodology. Thus, a hybrid parameterization scheme optimizes the structural topology and material orientation, and multi-component optimization (MTO)  decomposes product geometry while confirming manufacturing constraints that might significantly impact the quality and cost of the end product. Initially, a genetic algorithm was used to solve MTO \cite{Lyu2005} and then recently gradient optimization algorithms \cite{Zhou2018_MTO}. Zhou et al. \cite{Zhou2018} further extended their work for structures made of multiple composite components with tailored material orientations, without a prescribed set of alternative discrete angles. Therefore, this method can produce regions fabricated separately and then joined with either continuous or discrete material orientation methods.

Feature-based parametrization follows the ideology of ready-to-manufacturability with a necessary restriction on the spatial distribution of the fiber orientations. It envelops commercial aspects for the realizability of composite parts by introducing CAD-based features to ease the manufacturing process with the potential for layerwise design. Moreover, it further simplifies the design space by reducing the design variables considerably. It is noted that published work only considered stiffness-driven design. However, it is also critical to consider failure modes for composite parts manifested using the layerwise AM process. Incorporating these failure criteria renders markedly different designs that also raise the method's relevance in fabricating fiber-reinforced composite structures.

