\begin{frontmatter}

  \title{Topology Optimization strategies for continuous fiber-reinforced and functionally
    graded anisotropic composite structures: A brief review}

  %% use optional labels to link authors explicitly to addresses:
  \affiliation[label1]{organization={University of Bologna},
    addressline={Department of Industrial Engineering},
    city={Forlì},
    postcode={47121},
    state={Emilia-Romagna},
    country={Italy}}
  \author[label1]{Yogesh Gandhi \corref{cor1}}
  \cortext[cor1]{Corresponding author}
  \ead{yogesh.gandhi@unibo.it}
  \author[label1]{Giangiacomo Minak}
  \ead{giangiacomo.minak@unibo.it}

  \begin{abstract}

    Topology Optimization (TO) recently gained importance due to the development of Additive Manufacturing (AM) processes able to produce components with good mechanical properties. Among all types of additive manufacturing technologies, continuous fibre fused filament fabrication (CF4) can fabricate high-performance composites compared to those manufactured with conventional technologies. AM provides the excellent advantage of a very high degree of reconfigurability, which is in high demand to support the immediate short-term manufacturing chain in medical, transportation, and other industrial applications. Additionally, CF4 enables the fabrication of continuous fibre-reinforced composite (CFRC) materials and functionally graded anisotropic composite (FGRC) structures. The current version of AM allows us to integrate topology optimization strategies to design realizable fibre-reinforced composite (FRC) and FGRC structures for a given performance. Various TO strategies for attaining lightweight and high-performance designs have been proposed in the literature, which exploits the design freedom of AM. Therefore, this paper attempts to address works related to strategies employed to obtain optimal CFRC and FGRC structures. This paper intends to review, compare existing strategies, analyse their similarities and dissimilarities, and discuss challenges and future trends in this field.

  \end{abstract}
  %%Graphical abstract
  % \begin{graphicalabstract}
  %   %\includegraphics{grabs}
  % \end{graphicalabstract}

  %%Research highlights
  % \begin{highlights}
  %   \item Research highlight 1
  %   \item Research highlight 2
  % \end{highlights}

  \begin{keyword}
    %% keywords here, in the form: keyword \sep keyword

    %% PACS codes here, in the form: \PACS code \sep code

    %% MSC codes here, in the form: \MSC code \sep code
    %% or \MSC[2008] code \sep code (2000 is the default)
    topologoy optimization \sep fiber-reinforced composites \sep additive manufacturing \sep functionally graded material
  \end{keyword}

\end{frontmatter}