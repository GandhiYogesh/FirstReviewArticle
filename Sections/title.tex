\title{Topology Optimization strategies for continuous fiber-reinforced and functionally
graded anisotropic composite structures: A brief review}

%% use optional labels to link authors explicitly to addresses:
\author[label1]{Yogesh Gandhi \corref{cor1}}
\cortext[cor1]{Corresponding author}
\ead{yogesh.gandhi@unibo.it}
\affiliation[label1]{organization={University of Bologna},
addressline={Department of Industrial Engineering},
city={Forlì},
postcode={47121},
state={Emilia-Romagna},
country={Italy}}
\cortext[cor1]{Corresponding author}
\ead{yogesh.gandhi@unibo.it}

\author[label1]{Giangiacomo Minak}
\ead{giangiacomo.minak@unibo.it}

\begin{abstract}
    
    Among all types of Additive Manufacturing (AM) technology,Continuous Fiber
    Fused Filament Fabrication (CF4) can fabricate high-performance composites
    compared to those manufactured with conventional technologies. AM provides
    the excellent advantage of a very high degree of reconfigurability, which is
    in high demand to support  the immediate short-term manufacturing chain in
    medical, transportation, and other industrial applications. Additionally,
    the CF4 capability  enables the fabrication of Continuous Fiber-Reinforced
    Composite (CFRC) materials and Functionally GRaded Anisotropic Composite
    (FGRC) structures. The current expedition in AM allows us to integrate
    Topology Optimization (TO) strategies  to design realizable FRC and GFRC
    structures for a given performance.  Various TO strategies for attaining
    lightweight and high-performance designs have been proposed in the
    literature, which exploits AM's design freedom. Therefore, the paper
    attempts to address works related to TO strategies employed to  obtain
    optimal CFRC and FGRC structures. This review intends to overview, compare
    existing strategies, analyze their similarities and dissimilarities, and
    discuss challenges and future trends in this field.

\end{abstract}
%%Graphical abstract
\begin{graphicalabstract}
    %\includegraphics{grabs}
    \end{graphicalabstract}
    
    %%Research highlights
    \begin{highlights}
    \item Research highlight 1
    \item Research highlight 2
    \end{highlights}
    
    \begin{keyword}
    %% keywords here, in the form: keyword \sep keyword
    
    %% PACS codes here, in the form: \PACS code \sep code
    
    %% MSC codes here, in the form: \MSC code \sep code
    %% or \MSC[2008] code \sep code (2000 is the default)
    
    \end{keyword}
